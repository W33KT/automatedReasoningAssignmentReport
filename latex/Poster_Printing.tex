\documentclass{article}
\usepackage{amsmath, amssymb, graphicx, booktabs}
\usepackage[margin=2.5cm]{geometry}
\usepackage{listings}
\usepackage{xcolor}
\usepackage{float}

\title{Poster Printing using Z3}
\author{Luxue Wen\thanks{Student Number: 2271796 Email: l.wen@student.tue.nl}}
\date{September, 2025}

\lstset{
    basicstyle=\ttfamily\small,
    keywordstyle=\color{blue},
    commentstyle=\color{green!60!black},
    stringstyle=\color{red},
    numbers=left,
    numberstyle=\tiny,
    breaklines=true,
    frame=single,
    columns=fullflexible
}

\begin{document}

\maketitle

\section{Modeling the Problem}

\subsection{Parameters}
\begin{itemize}
  \item $N\_canvas$ = 3: number of each canvas.
  \item $N\_poster$ = 12: number of each canvas.
  \item $w = [5, 5, 4, 3, 7, 6, 5, 4, 6, 4, 6, 5]$: width of each poster.
  \item $h = [6, 6, 10, 11, 7, 10, 13, 10, 9, 15, 10, 10]$: height of each poster.
  \item $price = [10, 14, 13, 15, 10, 17, 21, 16, 16, 23, 19, 17]$: price of each poster.
  \item $W = [12, 12, 20]$: width of each canvas.
  \item $H = [12, 12, 20]$: height of each canvas.
  \item $cost = [30, 30, 90]$: cost of each canvas.
  \item $minimal\_profit$ = 60: minimal profit of printing.
\end{itemize}

\subsection{Decision Variables}
To fit posters into canvases, we introduce the following variables:
\begin{itemize}
  \item $z_{c,p} \in \mathbb{B}$ for $c = 1,...,N\_canvas,\ p = 1,...,N\_poster$: the value of $z_{c,p}$ will be true if and only if post[p] will be printed on canvas[c]
  \item $r_{c,p} \in \mathbb{B}$ for $c = 1,...,N\_canvas,\ p = 1,...,N\_poster$: the value of $r_{c,p}$ will be true if and only if post[p] will be turned $90^\circ$ 
  \item $u_c \in \mathbb{B}$ for $c = 1,...,N\_canvas$: the value of $u_c$ will be true if and only if canvas[c] will be used
  \item $x_{c,p},\ y_{c,p} \in \mathbb{N}$ for $c = 1,...,N\_canvas,\ p = 1,...,N\_poster$: the values of $x_{c,p}$ and $y_{c,p}$ indicate the bottom-left coordinate $(x,y)$ of $poster[p]$ placed in $canvas[c]$
  \item $w\_eff\_i,h\_eff\_i,w\_eff\_j,h\_eff\_j \in \mathbb{N}$: the width and height of post[i] and post[j]
  \item $total\_profit \in \mathbb{N}$: the value of total profit after printing
\end{itemize}

\subsection{Constraints}
For $post[p]$, it cannot be printed more than once. This is expressed by the formula
\[
\sum_i z_{p,i} \leq 1
\]
\\Next we determine that whether $post[p]$ fits into $canvas[c]$. This is expressed by the formula 
\begin{align*}
z_{c,p} \Rightarrow \, 
& \Bigl( (\lnot r_{c,p} \wedge x_{c,p} \ge 0 \wedge y_{c,p} \ge 0 \wedge x_{c,p} + w[p] \le W[c] \wedge y_{c,p} + h[p] \le H[c] \wedge w[p] \le W[c] \wedge h[p] \le H[c]) \nonumber\\
& \vee ( r_{c,p} \wedge x_{c,p} \ge 0 \wedge y_{c,p} \ge 0 \wedge x_{c,p} + w[p] \le W[c] \wedge y_{c,p} + h[p] \le H[c]) \wedge h[p] \le W[c] \wedge w[p] \le H[c] \Bigr)
\end{align*}
Additionally, every two posters $post[i]$ and $post[j]$ should have no overlap. This is expressed by the formula
\begin{align*}
(z_{c,i} \wedge z_{c,j}) \Rightarrow \, 
& \Bigl( (x_{c,i} + w_eff_i \le x_{c,j} ) \vee (x_{c,j} + w_eff_j \le x_{c,i} ) \nonumber\\
& \vee(y_{c,i} + h_eff_i \le y_{c,j} ) \vee (y_{c,j} + h_eff_j \le y_{c,i} ))
\end{align*}
Then, we associate $canvase[c]$ and $poster[p]$. This is expressed by the formula
\[
z_{c,p} \Rightarrow u_c
\]
Finally, we set the minimal profit. This is expressed by the formula
\[
total\_profit \ge mininal\_profit
\]

\subsection{Calculation Function}
The calculation function of $w\_eff\_i, \ h\_eff\_i, \ w\_eff\_j, \ h\_eff\_j$ is expressed by the formula
\[
w\_eff\_i =
\begin{cases}
h_i, & \text{if } r_{c,i} = 1 \\[4pt]
w_i, & \text{if } r_{c,i} = 0
\end{cases}
\ h\_eff\_i =
\begin{cases}
w_i, & \text{if } r_{c,i} = 1 \\[4pt]
h_i, & \text{if } r_{c,i} = 0
\end{cases}
\]
\[
w\_eff\_j =
\begin{cases}
h_j, & \text{if } r_{c,j} = 1 \\[4pt]
w_j, & \text{if } r_{c,j} = 0
\end{cases}
\ h\_eff\_j =
\begin{cases}
w_j, & \text{if } r_{c,j} = 1 \\[4pt]
h_j, & \text{if } r_{c,j} = 0
\end{cases}
\]
The calculation function of $total\_profit$ is expressed by the formula
\[
\sum_{\substack{c=1,\dots,N_{\text{canvas}} \\ p=1,\dots,N_{\text{poster}} \\ z_{c,p} = \text{true}}} price_p - \sum_{\substack{c=1,\dots,N_{\text{canvas}} \\ u_c = \text{true}}} cost_c
\]

\section{Solver Implementation using Z3}

\begin{itemize}
    \item In part (a), a solver instance s using s = Solver() is used to store and solve the constraints). 
    \item In part (b), an optimization solver instance using s = Optimize() is used to store constrains and handle objective functions to maximize variables.
    \item Constraints mentioned in 1.3 are added to the solver using s.add().
    \item The satisfiability of the constraints is checked by calling s.check().
    \item The specific values of the variables can be checked by calling s.model().evaluate
\end{itemize}

\section{Results Part}

\subsection{posters Assignments with three canvases}

\begin{table}[H]
\centering
\caption{posters Assignments with three canvases for Part (a)}
\small
\begin{tabular}{c|cccccccccccc|c|c|c}
\toprule
canvas & p0 & p1 & p2 & p3 & p4  & p5 & p6 & p7 & p8 & p9 & p10 & p11 & price & cost & profit\\
\midrule
0 & 0 & 0 & 0 & 0 & 0 & 0 & 0 & 0 & 0 & 0 & 0 & 0 & 0 & 0 &\\
1 & 10 & 14 & 0 & 15 & 0 & 0 & 0 & 16 & 0 & 0 & 0 & 0 & 55 & 30 &\\
2 & 0 & 0 & 13 & 0 & 0 & 17 & 21 & 0 & 16 & 23 & 19 & 17 & 126 & 90 &\\
\midrule
Total & & & & & & & & & & & & & 181 & 120 & 61\\
\bottomrule
\end{tabular}
\end{table}
It is possible to obtain profit at least 60.

\subsection{posters Assignments with two small canvases}

\begin{table}[H]
\centering
\caption{posters Assignments with three canvases for Part (a)}
\small
\begin{tabular}{c|cccccccccccc|c|c|c}
\toprule
canvas & p0 & p1 & p2 & p3 & p4  & p5 & p6 & p7 & p8 & p9 & p10 & p11 & price & cost & profit\\
\midrule
0 & 10 & 14 & 0 & 0 & 0 & 0 & 0 & 0 & 0 & 0 & 0 & 0 & 19 & 43 & 30\\
1 & 0 & 0 & 0 & 15 & 0 & 0 & 0 & 16 & 0 & 0 & 0 & 17 & 48 & 30 &\\
\midrule
Total & & & & & & & & & & & & & 91 & 60 & 31\\
\bottomrule
\end{tabular}
\end{table}
The highest profit created by the two small canvases is 31.

\subsection{Solver Performance and Optimization Results}

\begin{itemize}
    \item   
    In Part (a), the run time is around 900ms.
    \item 
    In Part (a), the run time is 2s.
\end{itemize}



\end{document}